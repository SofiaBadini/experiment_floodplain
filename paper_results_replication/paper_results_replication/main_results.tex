\section{Main results}\label{main-results}

\input{../../bld/tables/covariates}

\clearpage

\begin{figure}
     \centering
     \caption{Baseline beliefs vs. confidence}
     \begin{subfigure}[b]{\linewidth}
        \centering
         \includegraphics[width=\linewidth]{../../bld/figures/beliefs/beliefs1.png}
         \label{fig:risk}
     \end{subfigure}\\[-3ex]
     \begin{subfigure}[b]{\linewidth}
         \centering
         \includegraphics[width=\linewidth]{../../bld/figures/beliefs/beliefs2.png}
         \label{fig:damages}
     \end{subfigure}\\[-2ex]
         \begin{subfigure}[b]{\linewidth}
         \centering
         \includegraphics[width=\linewidth]{../../bld/figures/beliefs/beliefs3.png}
         \label{fig:comptot}
     \end{subfigure}\\[-2ex]
     \begin{subfigure}[b]{\linewidth}
        \centering
         \includegraphics[width=\linewidth]{../../bld/figures/beliefs/beliefs4.png}
         \label{fig:compshare}
     \end{subfigure}
    \label{fig:beliefs-summary}
\justifying
\footnotesize \textit{Notes}: Histogram of baseline beliefs (x-axis), elicited before the experimental treatments, versus confidence expressed in beliefs (y-axis). The blue bars indicate the share of respondents who gave the answer on the x-axis, while the grey bars indicate the average confidence expressed in the answer, again for every answer on the x-axis. The black lines represent 95\% confidence interval estimated via bootstrapping. The beliefs about damages are in thousand of Euros, and binned on 10k intervals for presentation purposes. The beliefs about 10-year flood probability are binned in 1pp intervals. 
\end{figure}

\clearpage

\begin{figure}
     \centering
     \caption{Information frictions}
     \begin{subfigure}[b]{\linewidth}
         \centering
        \caption{Information frictions vs. confidence}
         \includegraphics[width=\linewidth]{../../bld/figures/information/information1.png}
         \label{fig:aggregated}
     \end{subfigure}
     \hfill
     \begin{subfigure}[b]{\linewidth}
         \centering
         \caption{Disaggregated information frictions}
         \includegraphics[width=\linewidth]{../../bld/figures/information/information2.png}
         \label{fig:disaggregated}
     \end{subfigure}
    \label{fig:information-frictions}
\justifying
\footnotesize \textit{Notes}: Figure ~\ref{fig:aggregated} classifies respondents by number of incorrect answers to the information-based survey questions (left) and average confidence expressed in their answers (right). In figure~\ref{fig:disaggregated}, all answers to information-based questions are classified as correct or incorrect, grouped by topic, and plotted against the confidence level expressed by the respondent.
\end{figure}

\clearpage 

\input{../../bld/tables/frictions_maps}

\clearpage

\input{../../bld/tables/reading}

\clearpage

\input{../../bld/tables/outcomes_wtp}

\clearpage 

\begin{figure}[!t]
    \centering
    \caption{Hypothetical markets for flood insurance}
    \includegraphics[width=\linewidth]{../../bld/figures/wtp/wtp1.png}
    \label{fig:wtp}
\justifying
\footnotesize \textit{Notes}: This figures depicts the hypothetical markets for flood insurance under the "Neutral text" and "Risk profile" treatment arms. Respondents are categorized by whether they state a positive willingness-to-pay for insurance, and by objective flood risk category. The figure additionally includes average willingness-to-pay conditional on objective flood risk category. Of the total of survey respondents, 30\% are located in the 1 in 100 years floodplain, 44\% in the 1 in 1,000 years, and 26\% in the 1 in 10,000 years.  
\end{figure}

\clearpage

\begin{figure}
     \centering
     \caption{Belief updating over flood probability, by direction}
     \begin{subfigure}[b]{\linewidth}
         \centering
        \caption{Updates under "neutral text"}
         \includegraphics[width=\linewidth]{../../bld/figures/beliefs/beliefs5.PNG}
         \label{fig:updates-t1}
     \end{subfigure}
     \hfill
     \begin{subfigure}[b]{\linewidth}
         \centering
         \caption{Updates under "risk profile"}
         \includegraphics[width=\linewidth]{../../bld/figures/beliefs/beliefs6.PNG}
         \label{fig:updates-t2}
     \end{subfigure}
    \label{fig:updates}
\justifying
\footnotesize \textit{Notes}: Figure~\ref{fig:updates-t1} and ~\ref{fig:updates-t2} shows absolute size of belief updates by direction over flood probability for survey respondents in the "neutral text" and the "risk profile" treatment arm, respectively. The expected update direction is derived by comparing the respondents' information on their flood risk categories before the experiment with their true category, disclosed to them during the experiment. Respondents "read the text" if they spent at least 45 seconds on it.
\end{figure}

\clearpage